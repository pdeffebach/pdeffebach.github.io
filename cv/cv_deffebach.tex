\documentclass[12pt]{article}
% set the font
\usepackage[T1]{fontenc}
\usepackage{libertine}
\usepackage{inconsolata}

\usepackage{fancyhdr}

% set spacing for section titling
\usepackage{setspace}
% Make sure you can use accents
%\usepackage[spanish]{babel}
% set the margins
\usepackage[margin=1in]{geometry}
% Adjust title and section spacing
\usepackage{titling}
\usepackage{titlesec}
\setlength{\droptitle}{-10em}
% before and after sections
\titlespacing\section{0pt}{5pt}{2pt}
% Fix the spacing with itemize
\usepackage{enumitem}
\newenvironment{customitemize}
{ \begin{itemize}[
	leftmargin=\parindent,
	itemindent=-0.5\parindent,
	topsep = 0pt, 
	itemsep = -1pt, 
	label={}] }
{\end{itemize} }
\usepackage{bibentry}

\usepackage{graphicx}

\titleformat{\section}{\normalfont\fontsize{14}{1}\scshape\bfseries}{}{}{}[]

\pagestyle{fancy}

\fancyhead[HR]{\textbf{Peter Deffebach}}

\usepackage[dvipsnames]{xcolor}
\usepackage{hyperref}
\hypersetup{
  colorlinks   = true, 
  urlcolor     = Black,
  linkcolor    = Black, 
  citecolor   = Black 
}
% For my citations
\usepackage[round]{natbib}
\bibliographystyle{plainnat}
\usepackage{comment}
\begin{comment}
\addbibresource{cv_bib.bib}
\end{comment}

\usepackage{bibentry}
\nobibliography*

\begin{document}
\normalsize
\singlespacing

\thispagestyle{plain}


\newcommand{\link}[2]{{\color{blue}\href{#1}{#2}}}

\noindent{\Large{\textbf{\sc{Peter Deffebach}}}} \\
Boston University Department of Economics \\
270 Bay State Road, \\
Boston, MA 0221 \\
Email: \texttt{peterwd@bu.edu} \\
Cell: \texttt{503-853-6255} \\
Website: \href{https://pdeffebach.github.io/}{\texttt{pdeffebach.github.io}} \\
\rule{\textwidth}{1pt}
\section*{Education}

\begin{customitemize}
\item PhD, Economics, Boston University, Boston, MA  \hfill Expected 2025 
\begin{customitemize}
\item Dissertation title: Labor Market Churn, Development, and Quits: Evidence from Urban Ghana 
\item Main advisor: David Lagakos 
\item Dissertation Committee: David Lagakos, Yuhei Miyauchi, and Masao Fukui 
\end{customitemize}
\item AB, Economics, Princeton University, Princeton, NJ  \hfill 2017
\end{customitemize}
\section*{Fields of interest}

\begin{customitemize}
\item Macroeconomics of Development, Urban Economics, Labor Economics
\end{customitemize}


\section*{Publications}

\begin{customitemize}
	\item \bibentry{mexico}
\end{customitemize}

\section*{Working Papers}
\vspace{5pt}

\section*{Works in Progress}

\begin{customitemize}
	\item Quits in a Low-Income Urban Labor Market: Evidence from Ghana (Job Market Paper)
	\item Economic Development and the Spatial Distribution of Income in Cities, with David Lagakos, Yuhei Miyauchi, and Eiji Yamada
\end{customitemize}

\section*{Presentations}
\vspace{5pt}


\section*{Fellowships and Awards}


\begin{customitemize}
	\item International Growth Centre (IGC) Small Research Grant (\textsterling 20,000) \hfill 2022
	\item Structural Change and Economic Growth (STEG) Small Research Grant (\textsterling 15,000) \hfill 2023
	\item International Growth Centre (IGC) Full Research Grant (\textsterling 44,000) \hfill 2023
\end{customitemize}

\section*{Refereeing}


\begin{customitemize}
	\item Journal of Development Economics
	\item Review of Economic Dynamics
\end{customitemize}

\section*{Teaching Experience}

\begin{customitemize}
	\item Teaching Assistant, Introduction to Mathematical and Computational Economics (Graduate), Department of Economics, Boston University \hfill 2021, 2022
\end{customitemize}

\section*{Laguages}
\begin{customitemize}
\item English (native), Spanish (fluent)
\end{customitemize}

\section*{Computer skills}
\begin{customitemize}
\item Julia, R, Stata, GIS, Matlab, Python
\end{customitemize}

\section*{Citizenship}
\begin{customitemize}
\item USA
\end{customitemize}

\section*{References}
\begin{minipage}{0.3\textwidth}
\textbf{David Lagakos} \\
Department of Economiics,  \\
Boston University \\
Phone: \texttt{617-353-8903} \\
Email: \texttt{lagakos@bu.edu}
\end{minipage}
~
\begin{minipage}{0.3\textwidth}
\textbf{Yuhei Miyauchi} \\
Department of Economiics,  \\
Boston University \\
Phone: \texttt{617-353-5682} \\
Email: \texttt{miyauchi@bu.edu}
\end{minipage}
~
\begin{minipage}{0.3\textwidth}
\textbf{Masao Fukui} \\
Department of Economiics, \\
Boston University \\
Phone: \texttt{857-500-3712} \\
Email: \texttt{mfukui@bu.edu} 
\end{minipage}

\newpage
\thispagestyle{plain}
\noindent{\Large{\textbf{\sc{Peter Deffebach}}}} \\
\rule{\textwidth}{1pt}

\noindent \textbf{Quits in a Low-Income Urban Labor Market: Evidence from Ghana} (Job Market Paper)
\vspace{5pt}\\
\noindent Why are rates of wage employment so low in poor countries? I conduct a panel survey of job-seekers and a survey of firms in urban Ghana to explore labor market dynamics in depth. I document entry rates into employment are equal between the US and Ghana, but high exit rates mean Ghanaian job-seekers are only half as successful at finding wage work in the long run. In Ghana, I find exits are dominated by quits, while layoffs play a negligible role, in strong contrast with the USA, where layoffs dominate, and quits are infrequent. I examine, and reject, informational frictions as a key driver of high quit rates and show self-employment at most a moderate role. I show quits are most common among individuals who at Baseline are temporarily without flows of non-wage income. To quantify the contribution of changing non-wage income in driving quitting behavior, I build model of job search in which workers face uncertain non-wage income and accept and quit jobs to cope with temporary losses in income. I conclude 20 percent of the difference in exit rates between the USA and Ghana can be attributed to this mechanism.\\
\vspace*{5pt}


\noindent \textbf{Economic Development and the Spatial Distribution of Income in Cities} with David Lagakos, Yuhei Miyauchi, Eiji Yamada
\vspace{5pt}\\
\noindent We draw on new granular data from cities around the world to study how the spatial distribution of income within cities varies with development. We document that in less-developed countries, average incomes of urban residents decline monotonically in distance to the city center, whereas income-distance gradients are flat or increasing in developed economies. We also show that urban neighborhoods with natural amenities  -- in hills and near rivers -- are poorer than average in less-developed countries and richer than average in developed ones. We hypothesize that these patterns arise due to the differences in the provision of residential and transportation infrastructure within cites. Using a quantitative urban model, we show that observed differences in residential and transportation infrastructure help explain a significant fraction of how the spatial income distribution within cities varies with income per capita.

\nobibliography{cv_bib}

\end{document}